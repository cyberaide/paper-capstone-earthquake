\section{Other Machines}

\subsection{AMD5950X with RTX3070TI}
\label{sec:gcompute}


\begin{table}[htb]
  \caption{AMD5950X with RTX3070TI} \label{tab:gcompute}

  \resizebox{1.\columnwidth}{!}{
    \begin{tabular}{|ll|}
      \hline
 Attribute           & Value                                                            \\
 \hline
 \hline
OS & Ubuntu 20.04.3 LTS                                               \\
cpu                 & AMD Ryzen 9 5950X 16-Core Processor                              \\
cpu\_cores           & 16                                                               \\
cpu\_count           & 32                                                               \\
cpu\_threads         & 32                                                               \\
frequency           & scpufreq(current=2.2045, min=2200.0, max=3400.0)     \\
mem.total           & 128 GiB                                                          \\
python.pip          & 22.0.3                                                           \\
python.version      & 3.10.2                                                           \\
\hline
\end{tabular}
}
\end{table}

\begin{table}[htb]
  \caption{AMD5900HX with RTX3080} \label{tab:rcompute}

  \resizebox{1.\columnwidth}{!}{
    \begin{tabular}{|ll|}
      \hline
 Attribute           & Value                                          \\
 \hline
 \hline
OS  & Windows 11 Pro                                                  \\
cpu & AMD Ryzen 9 5900X 8-Core Processor                              \\
cpu\_cores          & 8                                               \\
cpu\_count          & 16                                              \\
cpu\_threads        & 16                                              \\
frequency           & scpufreq(current=3600.0, min=2700.0, max=4800.0)\\
mem.total           & 32 GiB                                          \\
python.pip          & 22.0.3                                          \\
python.version      & 3.10.2                                          \\
\hline
\end{tabular}
}
\end{table}
