\PassOptionsToPackage{hyphens}{url}
\documentclass[sigplan,screen]{acmart}
\usepackage{todonotes}
\usepackage{graphicx}
\usepackage{wrapfig}
\usepackage[pdf]{graphviz}
\usepackage{xpatch}
\usepackage{adjustbox}
\usepackage{makecell} 
\usepackage{ifthen}

\newcommand{\TODO}[1]{\todo[inline]{#1}}
%\providecommand{\mlperf}{MLPerf\texttrademark{}}
%\providecommand{\mlcube}{MLCube\texttrademark{}}
\providecommand{\mlperf}{MLPerf}
\providecommand{\mlcube}{MLCube}

\newcommand{\BIOFIG}[3]{%
\parindent 0pt%
\setlength{\intextsep}{6pt}%
\setlength{\columnsep}{6pt}%

{\bf #2}

\begin{wrapfigure}{r}{0.2\columnwidth}%
    \includegraphics[width=0.2\columnwidth]{#1}%
\end{wrapfigure}%
#3%
}


\setcopyright{none}
%\setcopyright{rightsretained}
\copyrightyear{}
%\acmYear{2022}
\acmDOI{}

\acmConference[Data Science Capstone, University of Virginia,
  2022]{MLCommons Earthquake Science Benchmark}{Jan. 22 -- Aug. 31
  2022}{Charlotsville, VA, USA}

\acmBooktitle{Woodstock '18:
  ACM Symposium on Neural Gaze Detection, June 03--05, 2018,
  Woodstock, NY}

\acmPrice{}
\acmISBN{}
%%\acmSubmissionID{123-A56-BU3}
%%\citestyle{acmauthoryear}

\makeatletter
\newcommand*{\addFileDependency}[1]{% argument=file name and extension
  \typeout{(#1)}
  \@addtofilelist{#1}
  \IfFileExists{#1}{}{\typeout{No file #1.}}
}
\makeatother
\xpretocmd{\digraph}{\addFileDependency{#2.dot}}{}{}


\newcommand{\GITISSUE}[1]{\href{https://github.com/Data-ScienceHub/mlcommons-science/issues/#1}{GitHub Issue #1}}
\newcommand{\GITPULL}[1]{\href{https://github.com/laszewsk/mlcommons/pull/#1}{GitHub Pull #1}}

\begin{document}


\title{MLCommons Earthquake Science Benchmark}

\author{Thomas Butler}
\orcid{0000-0003-1515-0098}
\affiliation{%
  \institution{University of Virginia}
  \city{Charoletteville}
  \country{USA}}
 \email{Tommysbutler@gmail.com}

\author{Robert Knuuti}
\orcid{0000-0003-2614-0899}
\affiliation{%
  \institution{University of Virginia}
  \city{Charoletteville}
  \country{USA}}
\email{robert.knuuti@gmail.com}

\author{Jake Kolessar}
\affiliation{%
  \institution{University of Virginia}
  \city{Charlottesville}
  \state{VA}
  \postcode{22911}
  \country{United States}}
\email{jakekolessar@gmail.com}

\author{Geoffrey C. Fox}
\affiliation{%
  \institution{University of Virginia}
  \streetaddress{Biocomplexity Institute
Town Center Four
994 Research Park Boulevard}
  \city{Charlottesville}    
  \state{VA}
  \postcode{22911}
  \country{USA}}
\email{gcfexchange@gmail.com}

\author{Gregor von Laszewski}
\orcid{0000-0001-9558-179X}
\affiliation{%
  \institution{University of Virginia}
  \streetaddress{Biocomplexity Institute
Town Center Four
994 Research Park Boulevard}
  \city{Charlottesville}    
  \state{VA}
  \postcode{22911}
  \country{USA}}
\email{laszewski@gmail.com}

\author{Judy Fox}
\affiliation{%
  \institution{University of Virginia}
  \city{Charlottesville}      
  \state{VA}
  \postcode{22911}
  \country{USA}}
\email{cwk9mp@virginia.edu}

\renewcommand{\shortauthors}{Butler, Knuuti, Kolesar, G Fox, G von Laszewski}

\begin{abstract}
TBD
\end{abstract}



\begin{CCSXML}
<ccs2012>
 <concept>
  <concept_id>10010520.10010553.10010562</concept_id>
  <concept_desc>Computer systems organization~Embedded systems</concept_desc>
  <concept_significance>500</concept_significance>
 </concept>
 <concept>
  <concept_id>10010520.10010575.10010755</concept_id>
  <concept_desc>Computer systems organization~Redundancy</concept_desc>
  <concept_significance>300</concept_significance>
 </concept>
 <concept>
  <concept_id>10010520.10010553.10010554</concept_id>
  <concept_desc>Computer systems organization~Robotics</concept_desc>
  <concept_significance>100</concept_significance>
 </concept>
 <concept>
  <concept_id>10003033.10003083.10003095</concept_id>
  <concept_desc>Networks~Network reliability</concept_desc>
  <concept_significance>100</concept_significance>
 </concept>
</ccs2012>
\end{CCSXML}

\ccsdesc[500]{Computer systems organization~Embedded systems}
\ccsdesc[300]{Computer systems organization~Redundancy}
\ccsdesc{Computer systems organization~Robotics}
\ccsdesc[100]{Networks~Network reliability}

%%
%% Keywords. The author(s) should pick words that accurately describe
%% the work being presented. Separate the keywords with commas.
\keywords{MLCommons, Science AI Benchmark, Earthquake, Deep Learning}

\maketitle

\TODO{Fix the classification. See  \url{see http://dl.acm.org/ccs.cfm}} 


\section{Introduction}

Several research efforts have been executed over the years to
establish a complex set of factors that accurately predict where an
earthquake could occur, and what impacts an earthquake would cause to
the environment when they occur.  Not only is this an algorithmically
complex problem, but it is also computationally dense and requires
high performance hardware to obtain these
predictions\cite{fox2022aiforscience}.  The intersection of these two
elements create an ideal environment for building fair benchmarks.  By
establishing a rigorous process for constructing and running deep
learning models the basis for comparing not just computational power
in the High Performance Computing ecosystem, but also establish a
benchmarks of new scientific discoveries from its execution.

For this study, we review the following model architectures across
various hardware platforms to establish this benchmark:

\begin{itemize}
    \item Long Short-Term Memory (LSTM) Recurrent Neural Network (RNN)
    \item Spacio-Temporal Transformer Network (S2TN) as a Science Transformer
    \item Temporal Fusion Transformer (TFT)
\end{itemize}

All three models leverage Deep Learning algorithms based upon transformer models.
For each model, the benchmark analysis across the following deep learning frameworks:

\begin{itemize}
    \item Tensorflow (original reference)
    \item Keras (original reference)
    \item MXNet
    \item Pytorch.
\end{itemize}

Each framework have their own performance measures that will shape the
reference benchmark implementation.

\subsection{MLCommons}

MLCommons is an open engineering organization that seeks to unify the
engineering and machine learning communities through system
benchmarks, open data, and best practices \cite{www-mlcommons}.  By
using open data, a reference implementation for benchmarking, and
executing best practices, the necessary rigor enables the construction
repeatable research to establish a basis when performing future
analysis.  Using MLCommons organizational standards set by \mlperf{},
the dimensions of a the minimally viable solution for benchmarking are
know and to establish our comparison criteria across our models and
the executions on the different hardware sets\cite{mattson2019mlperf}.

\TODO{Need to confirm what the criteria of the benchmark submission is.}

\TODO{Address measure of Scientific Discovery, which is unique to the science benchmarks.}

\subsection{Why Build a Science Benchmark?}

The \mlperf{} framework already have several feature complete
benchmarks that can judge system performance, however these benchmarks
are centered on well known algorithms with solutions that are well
known.  Additionally, some CPU architectures create optimized
instructions to solve common algorithmic problems at the microcode
level, creating a performance boost when these prepared instructions
are encountered\cite{cheong1997optimizer}.  While these measures are
still of great importance, they do not reflect the performance of
systems based on non-deterministic measures that exist in nature.
This makes benchmarking using a scientific dataset an ideal candidate
for calculating performance, as there's an inherit randomness that's
difficult to "game" this system and build predictions.  This is
especially true when attempting to predict the next occurrence of an
earthquake, as complex models are required to gain predictive
insights\cite{www-murakami2021}.

\subsection{What is MLCommons Science Research?}

MLCommons Science Research group is involved in edge and data center
issues, end-to-end systems, inference and training. MLCommons Science
Research focuses on Science applications and not industry. There are
similarities between industry and science, for example image data is
similar but some algorithms can be very different. The best example of
this is particle physics experiments, quarks and lumens are not a
common thing to model in industry \cite{www-mlcommons-science}.

The primary goal of MLCommons Science Research is to develop optimal
methods to improve both speed and efficiency of scientific
outcomes. This is done by measure how different models perform on
different scientific datasets, in particular score the models based on
how well they do on scientific discovery. This could be anywhere from
accuracy of a model to time to reach specific accuracy target when new
data is introduced. In general, how well each model matches the goals
of that particular scientific field. By using verified datasets
following best standardized practice on how to build and exchange
models using the reference materials on benchmarked machines,
knowledge can flow more easily and freely between various
researchers. By having these standard methods, both the monetary cost
and time can be reduced between researcher regardless of their
speciality.

Currently MLCommons Science Research has four datasets based on
climate, material science, medicine and earthquakes with a fifth
dataset on plasma physics being added soon. Each dataset has their own
method used to solve their scientific problem
\cite{www-mlcommons-science}.

\subsection{What is \mlcube{}?}

\mlcube{} is a set of standards and software developed by MLCommons
for delivering machine learning models across container runtimes such
as docker, kubernetes, singularity, or for automating the model's
execution over ssh.  \mlcube{} provides a common configuration format
and interface such that consumers of a model do not need to worry
about the underlying configuration and baseline of supporting
libraries, and enables the user to focus on the model
itself\cite{www-mlcube}.

There are two targets when building \mlcube{} models:

\begin{itemize}
  
    \item \textbf{Developer.}  The developer produces a YAML
      configuration file that describes the target execution details
      including platform, runtime dependencies, inputs, outputs,
      metadata about the model, and the source code.  Additionally,
      this configuration file offers branching paths based on what
      \mlcube{} runtime backend is selected (for example, the
      container image to use for docker can be configured differently
      than a kubeflow configuration).  This is so the implementation
      details can be explicitly addressed, if necessary, by the model
      developer and abstracts this complexity away from consumers.

    \item \textbf{User.}  The user of the system will receive the
      \mlcube{} code repository and execute a command that identifies
      which platform they intend to run the model on and any type of
      input parameters to the model detailed by the developer.
\end{itemize}

As described by the \mlcube{} development team, this ecosystem
establishes a contract between model developers and consumers of that
model so that results can be easily shared and provide a framework to
produce consistent results.

The MLCommons Earthquake Science Benchmark models target the \mlcube{}
mechanism for delivery.

\subsection{Earthquake prediction}

Earthquake prediction is a complex and old problem. There are two
major tasks related to earthquakes. One is forecasting earthquake
occurrences/likelihoods. Another is to predict damage once an
earthquake does happen. The first is difficult because the details of
the underground plates and the friction laws between them are not
known and furthermore the earthquake causes phase transitions between
plates which makes for unpredictable movement. The second, we need to
predict earthquake energy wave movements. Since both of these problems
are so complex with many hidden variables and the equations governing
the phenomenon are unknown or incomplete, deep learning can be use to
learn the various hidden patterns in the data which the model can then
use to forecast earthquakes into the future. \cite{fox2021earthquake}

\subsection{Energy-based averaging, the log loss equation}

\TODO{This section is in progress! Enter Energy Weighted Quantities
  equations picture below}

This function maximizes the largest earthquake energy values for the 2
week period.  Magnitude averaged over the bin All input values and
predictions are independently normalized and have a maximum modulus of
1 over all space and time values. \cite{fox2022aiforscience}


\subsection{Earthquake Data turned into Spatial bag Time Series}

\TODO{This section is in progress!}

The data being used is the Southern California location data, between
32 to 36 lat -120 to -114 long. The data is recorded between
1950-2020. Each location point is is 0.1 by 0.1 degrees or ll km by ll
km in size or 2400 points in total. The points are then put into
spatial bag which are based on a bunch of different location points,
necessarily nearby spatially. Each location point has a time series
associated with it. The time series data holds magnitude, depth,
spatial position, and time. \cite{fox2021earthquake}

The data was accumulated into 2 week samples using energy-based
averaging, the log loss equation.


\TODO{provide introduction to earthquake}

\TODO{Review how these papers contribute to introduction}

\begin{itemize}
    \item Review IEEE Presentation \cite{fox2022aiforscience}
    \item Review Fox paper \cite{fox2021earthquake}
\end{itemize}

\section{Timeseries Earthquake Prediction}

\subsection{LSTM}

\begin{figure}[htb]
\resizebox{0.75\columnwidth}{!}{
    \digraph{lstmarchitecture}{
        node [shape=box style=rounded color=grey];
        Embedder -> "2-Layer LSTM" -> "Output Mapper";
    }
}
\caption{LSTM Model}\label{fig:lstm-architecture}
\end{figure}


\subsection{TFT}

\TODO{Fix this figure}

\begin{figure}[htb]
\resizebox{0.75\columnwidth}{!}{
%    \digraph{tftarchitecture}{
%        compound=true;
%        newrank=true;
%        rankdir="LR";
%        node [shape=box style=rounded color=grey];
        
%        start[label="Embedders"];
%        ascbackward[label="2-layer LSTM as Backward encoder"];
%        ascforward[label="2-layer LSTM as Forward decoder"];
%        ta[label="Temporal Attention"];


 %       start -> ta [lhead="clusterstaticcontext", constraint=false];
 %       ta -> "Output Mappers";
 %       
 %       subgraph clusterstaticcontext {
 %           rankdir="TB";
 %           label="Static Context";
 %           {rank="same"; ascbackward; ascforward;}
 %           ascbackward -> ta;
 %           ascbackward -> ascforward;
 %           ascforward -> ta;
%        }
%    }
}
\caption{TFT Model}\label{fig:tft-architecture}
\end{figure}

%% Toying with tikz for generating the grpahic.
%\tikz [nodes={draw}] {
%    \node (a) {Embedders};
%    \node (b) [below left of=a] {2-layer LSTM as Backward Encoder};
%    \node (c) [below right of=a] {2-layer LSTM as Forward Decoder};
%    \node (d) [below right of=b] {Temporal Attention};
%    \node (e) [below of=d] {Output Mappers};
%    \draw[->] (a) -- (b);
%    \draw[->] (a) -- (c);
%    \draw[->] (b) -- (c);
%    \draw[->] (b) -- (d);
%    \draw[->] (c) -- (d);
%    \draw[->] (d) -- (e);
%}



%\begin{figure}[h]
%  \centering
%  \includegraphics[width=\linewidth]{sample-franklin}
%  \caption{1907 Franklin Model D roadster. Photograph by Harris \&
%    Ewing, Inc. [Public domain], via Wikimedia
%    Commons. (\url{https://goo.gl/VLCRBB}).}
%  \Description{A woman and a girl in white dresses sit in an open car.}
% \end{figure}

\section{Benchmarks}


\subsection{Gregor's MNIST}

Gregors MNIST benchmark results are summarized in Table~\ref{tab:mnist}.

\begin{table*}[!ht]
\caption{Gregors MNIST Benchmarks}\label{tab:mnist}
    \centering
      \begin{adjustbox}{max width=\textwidth}
    \begin{tabular}{|l|l|l|l|l|l|l|l|l|l|}
    \hline
        Machine & real & user & sys & Driver & CUDA & GPU & \makecell{Date CPU \\released} & CPU & \makecell{Date CPU \\released} \\ \hline
        Gregors Machine & 0m11.534s & 0m13.914s & 0m05.186s & 510.47.03 & 11.6 & Gigabyte RTX3070 TI & May 31, 2021 & AMD 5950X & Nov 2020 \\ \hline
        Fox DGX & 0m19.987s & 5m12.991s & 0m49.266s & 450.142.00 & 11.0 & NVIDIA A100 80GB & & AMD EPYC 7742 64-Core & Aug 2019 \\ \hline
        Rivanna A100 & 0m29.263s & 0m14.585s & 0m7.399ss & 470.82.01 & 11.4 & NVIDIA A100-SXM4-40GB & May 14, 2020 & Intel(R) Xeon(R) CPU E5-2630 v3 @ 2.40GHz & Q3  2014 \\ \hline
        MacBook Pro & 0m31.01s & 0m25.26s & 130\% & N/A & N/A & N/A & N/A & M1 Max 66GB & Nov 2021 \\ \hline
        Rivanna P100 & 0m35.732s & 0m17.253s & 0m7.595s & 470.82.01 & 11.4 & Tesla P100-PCIE & & Intel(R) Xeon(R) CPU E5-2630 v3 @ 2.40GHz & Q3  2014 \\ \hline
        Rivanna V100 & 0m43.160s & 0m15.510s & 0m6.894s & 470.82.01 & 11.4 & Tesla V100-SXM2 & & Intel(R) Xeon(R) CPU E5-2630 v3 @ 2.40GHz & Q3  2014 \\ \hline
        Rivanna K80 & 0m57.588s & 0m20.322s & 0m9.612s & 470.82.01 & 11.4 & NVIDIA TESLA K80 & & Intel(R) Xeon(R) CPU E5-2630 v3 @ 2.40GHz & Q3  2014 \\ \hline
        Rivanna Frontend & 1m11.535s & 1m00.780s & 0m10.352s & N/A & N/A & N/A & & Intel(R) Xeon(R) CPU E5-2630 v3 @ 2.40GHz & Q3  2014 \\ \hline
    \end{tabular}
    \end{adjustbox}
\end{table*}

\subsection{Hardware}

\subsubsection{Aurora}
\label{id:anl} 

The Aurora supercomputer \cite{www-aurora} is locatd at that Argonne
Leadership Computing Facility (ACLF) part of the U.S. Department of
Energy (DOE).  It was developed with the help of industry experts from
Intel and Cray. Aurora offers HPC for data analytics and AI
development with a peak performance of over 2 Exaflop DP. The system
is based on the HPE Cray EX supercomputer platform and offers over
9,000 compute nodes.

\subsubsection{Summit}
\label{id:ornl} 

The Summit supercomputer\cite{www-summit} is located in the Oak Ridge
Leadership Computing Facility (OLCF) part of the U.S. Department of
Energy (DOE). The system is avalable to government, academic, and
industry researchers to solve complex problems in all fields of
science. The system is powered by 4,608 nodes which each house
multiple IBM POWER9 CPUs and NVIDIA Volta GPUs on top of half a
terabyte of coherent memory.


\subsubsection{Perlmutter}
\label{id:nersc} 

The Perlmutter \cite{www-perlmutter} supercomputer is located at the
National Energy Research Scientific Computing Center (NERSC) which is
part of the U.S. Department of Energy (DOE).  It has a 12 cabinet GPU
systems with over 1,500 nodes and a 12 cabinet CPU system containing
over 3,000 nodes \TODO{This is inconsistant or not properly explained
  as 2 node numbers are used}. It is supported by 35 petabytes of
all-flash storage which can move data at 5 terabytes/sec.


\subsubsection{Expanse}
\label{id:sdsc}

The Expanse \cite{www-expanse} system at the Sandiego Super Computing
Center (SDSC) ....

\TODO{Expanse description incomplete}

\subsubsection{Rivanna}
\label{id:uva} 


University of Virginia Rivanna
  
\TODO{Rivann description is inomplete}

UVA, Rivanna, A100, RTX3090, others may not have enough memory.
\cite{www-rivanna} \TODO{description of Rivanna on web page is not
  accurate.}

Containers are available on Rivanna through Singularity
\cite{www-singularity} that can be loaded with modules
\cite{www-modules}. We plan to utilize singularity, but target the
formulation of the container in docker and than transform it to
singularity. As singularity is installed also on other machines, we
will leverage this deployment also on them.

\begin{figure}[htb]
\resizebox{0.75\columnwidth}{!}{
\digraph{singularity}{
    node [shape=box style=rounded color=grey];
    Application -> GitHub
    "Dockerfile" -> "Docker Container" -> "Singularity Conversion" -> "Singularity Container" -> "Batch job (SLURM)"
}
}
\caption{Containerized Science Applications in Singularity}
\end{figure}

\subsection{Pearl}
\label{id:ral} 

The PEARL \cite{www-pearl-1} system is located t the Alan Turing
Institute in the the Rutherford Appleton Laboratory, UK.  It was
developed for the advancement of AI and Machine Learning research with
large-scale scientific datasets, already contributing to the fields of
materials, environmental and life sciences, and astronomy. It is
comprised of two NVIDIA DGX2 nodes, 32 GPUs, 3TB of system memory, 1TB
of GPU memory, and 600TB of high-speed storage.


\subsection {Nvidia DGX Workstation}
\label{id:dgx} 

The NVIDIA DGX Station A100 \cite{www-dgx-station-a100} is a
commercially available product that can provide powerful computing
resources to all industries. The DGX Station is an AI resource that
can run parallel jobs and support training, inference, and data
analytics. It contains 4 NVIDIA A100 GPUs and up to 320 GBs of GPU
memory. With the use of Multi-Instance GPU (MIG), each A100 GPU can be
partitioned into 7 isolated instances.


%%%%%%%%%%%%%%%%%%%%%%%%%%%%%%%%%%%%%%%%%%%%%%%%%%%%%%%%%%%%%%%%%%%%%%%%%%%%%%%
%
% THIS SECTION MAY NOT BE COPIED INTO ANOTHER PAPER WITHOUT MAKING GREGOR VON LASZEWSKI COAUTHOR
%
%
\newcommand{\nvidia}[1]{
\ifthenelse{\equal{#1}{A100}}{\href{https://www.nvidia.com/en-us/data-center/a100/}{NVIDIA #1}}{}%
\ifthenelse{\equal{#1}{V100}}{\href{https://www.nvidia.com/en-us/data-center/a100/}{NVIDIA #1}}{}%
\ifthenelse{\equal{#1}{K80}}{\href{https://www.nvidia.com/en-gb/data-center/tesla-k80/}{NVIDIA #1}}{}%
\ifthenelse{\equal{#1}{P100}}{\href{https://www.nvidia.com/en-us/data-center/tesla-p100/}{NVIDIA #1}}{}%
\ifthenelse{\equal{#1}{RTX3090}}{\href{https://www.nvidia.com/en-us/geforce/graphics-cards/30-series/rtx-3090/}{NVIDIA #1}}{}%
\ifthenelse{\equal{#1}{RTX2080TI}}{\href{https://www.nvidia.com/en-eu/geforce/20-series/}{NVIDIA #1}}{}%
\ifthenelse{\equal{#1}{Volta}}{\href{https://www.nvidia.com/en-us/data-center/volta-gpu-architecture/}{NVIDIA #1}}{}%
}
%\newcommand{\nvidiaV}{\href{https://www.nvidia.com/en-us/data-center/v100/}{NVIDIA V100}}

\begin{table*}[htb]
    \caption{Overview of compute resources.}
    \label{tab:hwoverview}
    \centering
%\begin{adjustbox}{angle=90}
\begin{adjustbox}{max width=\textwidth}
    \begin{tabular}{|r|l|ll|r|l|l|l|l|l|}
        \hline
        See also & Organization   & Machine                         & & Processors    & GPUs             & \makecell{Memory\\/Device} &  \makecell{GPU\\/Device} & \makecell{No. of\\ nodes} & Commissioned \\ 
        \hline
        \hline
         \ref{id:anl}   & ANL    & Aurora & \cite{www-aurora}         &              &                  & & & &    ??? 2022   \\ \hline
         \ref{id:ornl}  & ORNL   & Summit &\cite{www-summit}         &        27000 & \nvidia{volta}     & & & &               \\ \hline
         \ref{id:nersc} & NERSC  & Perlmutter & \cite{www-perlmutter} &         6000 & \nvidia{A100}      & & & & Jan 2022      \\ \hline
         \ref{id:sdsc}  & SDSC   & Expanse & \cite{www-expanse}       &          200 & \nvidia{V100}
         SMX2 & & & &               \\ \hline
         \ref{id:uva}   & UVA    & Rivanna &\cite{www-rivanna}       &            8 & \nvidia{A100}      & 80GB & ? & &  Feb 2022      \\
            &    &  &       &            8 & \nvidia{A100}      & 40GB & 8 & &       \\
            &    &  &       &            ? & \nvidia{RTX3090}   & 24GB & ? & ? & 2021          \\
            &    &  &       &            ? & \nvidia{K80}       & 11GB & 8 & 9 & 2021          \\
            &    &  &       &            ? & \nvidia{V100}      & 16GB & 4 & 1 & 2021          \\
            &    &  &       &            ? & \nvidia{V100}      & 32GB & 4 & 12 & 2021          \\
            &    &  &       &            ? & \nvidia{P100}      & 12GB & 4 & 3 & 2021          \\
            &    &  &       &            ? & \nvidia{RTX2080TI} & 11GB & 10 & 2 & 2021          \\
         \hline
         \ref{id:ral}   & RAL    & Pearl &\cite{www-pearl-1}         &           16 & \nvidia{V100}      & & & &  \\ \hline
         \ref{id:dgx}   & Fox    & DGX Station A100 &\cite{www-dgx-station-a100} & 4 & \nvidia{A100}      & 80GB & 4 & 1 & May 2021      \\
         \hline
    \end{tabular}
    \end{adjustbox}
\end{table*}

Rivanna: \url{https://www.rc.virginia.edu/userinfo/rivanna/overview/}


\TODO{Add your personal computers to this table \url{https://github.com/Data-ScienceHub/mlcommons-science/issues/37}}
\begin{table*}[htb]
    \caption{Overview of compute resources.}
    \label{tab:desktops}
    \centering
%\begin{adjustbox}{angle=90}
\begin{adjustbox}{max width=\textwidth}
    \begin{tabular}{|r|l|l|r|l|l|l|l|l|}
        \hline
        See also & Organization   & Machine                         & Processors    & GPUs             & \makecell{Memory\\/Device} &  \makecell{GPU\\/Device} & \makecell{No. of\\ nodes} & Commissioned \\ 
        \hline
        \hline
         Table \ref{tab:gcompute}   & 
         Gregor    & 
         5950X   &  
         1 &   
         \href{https://www.gigabyte.com/Graphics-Card/GV-N307TGAMING-OC-8GD#kf}{GIGABYTE Gaming RTX 3070TI}  & 
         8GB & 
         1 & 
         1 & 
         Sep 2021   \\
            & 
            & 
            &  
            &   
         \href{https://rog.asus.com/us/graphics-cards/graphics-cards/rog-strix/rog-strix-rtx3090-o24g-gaming-model/}{ASUS ROG Strix RTX 3090} & 
         24GB & 
         1 & 
         1 & 
         Feb 2022   \\
         \hline
         \ref{sec:tcomputer}   & Thomas.A  & i7-8750H &  1 &   NVIDIA GeForce RTX 2070 with Max-Q Design  & 16GB & 1 & 1 & 2019   \\
         \hline
         \ref{sec:t2computer}   & Thomas.B  & i7-7700HQ &  1 &   NVIDIA GeForce GTX 1060  & 16GB & 1 & 1 & 2017   \\
         \hline
    \end{tabular}
    \end{adjustbox}
\end{table*}

\section{Other Machines}

\subsection{AMD5950X with RTX3070TI}
\label{sec:gcompute}


\begin{table}[htb]
  \caption{AMD5950X with RTX3070TI} \label{tab:gcompute}

  \resizebox{1.\columnwidth}{!}{
    \begin{tabular}{|ll|}
      \hline
 Attribute           & Value                                                            \\
 \hline
 \hline
OS & Ubuntu 20.04.3 LTS                                               \\
cpu                 & AMD Ryzen 9 5950X 16-Core Processor                              \\
cpu\_cores           & 16                                                               \\
cpu\_count           & 32                                                               \\
cpu\_threads         & 32                                                               \\
frequency           & scpufreq(current=2.2045, min=2200.0, max=3400.0)     \\
mem.total           & 128 GiB                                                          \\
python.pip          & 22.0.3                                                           \\
python.version      & 3.10.2                                                           \\
\hline
\end{tabular}
}
\end{table}


\begin{acks}
\TODO{Geoffrey: please add acknowledgement for funding, other acks}
\end{acks}

%
% THIS SECTION MAY NOT BE COPIED INTO ANOTHER PAPER WITHOUT MAKING GREGOR VON LASZEWSKI COAUTHOR
%
%%%%%%%%%%%%%%%%%%%%%%%%%%%%%%%%%%%%%%%%%%%%%%%%%%%%%%%%%%%%%%%%%%%%%%%%%%%%%%%

\section{Other Machines}

\subsection{AMD5950X with RTX3070TI}
\label{sec:gcompute}


\begin{table}[htb]
  \caption{AMD5950X with RTX3070TI} \label{tab:gcompute}

  \resizebox{1.\columnwidth}{!}{
    \begin{tabular}{|ll|}
      \hline
 Attribute           & Value                                                            \\
 \hline
 \hline
OS & Ubuntu 20.04.3 LTS                                               \\
cpu                 & AMD Ryzen 9 5950X 16-Core Processor                              \\
cpu\_cores           & 16                                                               \\
cpu\_count           & 32                                                               \\
cpu\_threads         & 32                                                               \\
frequency           & scpufreq(current=2.2045, min=2200.0, max=3400.0)     \\
mem.total           & 128 GiB                                                          \\
python.pip          & 22.0.3                                                           \\
python.version      & 3.10.2                                                           \\
\hline
\end{tabular}
}
\end{table}



\bibliographystyle{ACM-Reference-Format}
\bibliography{paper-capstone-earthquake}

\section*{Biographies}

\BIOFIG{images/bio/gregor.png}{Thomas Butler}{is a Graduate Student at
  University of Virginia's School of Data Science. His undergraduate
  degree is in Biomedical engineering. He has over eight years of
  experience in the Infertility field helping patients, jointly
  running a Andrology lab, and contributing research to advance the
  field through joint research on how AMH effects pregnancy outcomes
  and sperm antibodies effect PSA. He has a certificate in Data
  Analytics from Georgia Institute of Technology.}

\BIOFIG{images/bio/gregor.png}{Robert Knuuti}{is a Graduate Student at
  University of Virginia's School of Data Science. He has over 10
  years experience in system architecture and software engineering,
  and specializes in Development Operations and Cloud Computing. He
  has constructed air gapped Continuous Integration and Continuous
  Delivery systems for multiple organizations each supporting more
  than 100 developers and has facilitated the construction of
  repeatable, tractable builds for users of these systems.}

\BIOFIG{images/bio/gregor.png}{Jake Kolessar}{is a Graduate Student at
  the University of Virginia's School of Data Science. He has a
  background in mechanical engineering and 2 years of experience as a
  Modeling, Simulation and Analysis Engineer. He has supported the
  software design and development of modeling capabilities for event
  simulation products as well as the integration of models into the
  simulation framework.}

\BIOFIG{images/bio/fox.png}{Geoffrey C. Fox}{TBD TBD TBD TBD TBD
  TBD TBD TBD TBD TBD TBD TBD TBD TBD TBD TBD TBD TBD TBD TBD TBD TBD
  TBD TBD TBD TBD TBD TBD TBD TBD TBD TBD TBD TBD TBD TBD TBD TBD TBD
  TBD TBD TBD TBD TBD TBD TBD TBD TBD TBD TBD TBD TBD TBD TBD TBD TBD
  TBD TBD TBD TBD TBD TBD TBD TBD}

%\begin{wrapfigure}{r}{0.25\columnwidth}
%\includegraphics[width=0.24\columnwidth]{images/bio/gregor.png}
%\end{wrapfigure}


\BIOFIG{images/bio/gregor.png}{Gregor von Laszewski}{is a Research
  Professor at University of Virginia.  He has more than 30 years of
  experience in parallel and distributed computing.  Selected research
  organizations he worked at include NASA, Argonne National
  Laboratory, Indiana University. He is proud to have been
  contributing members of research teams that invented hybrid parallel
  genetic algorithms, Grid computing, and the first large scale
  academic hybrid cloud, as well as the larges scientific bibliometric
  analysis of XSEDE in the world.}



\appendix

\section{Manuals Developed}

The following manuals have been developed by the team:

\begin{itemize}
    \item Introduction to Python \cite{las-intro-python}
\end{itemize}

\TODO{There are more we developed in the web page} 

\clearpage 

\section{Todo}

\subsection{Manual List}
\section{TODO and integrate}

Things to do: 

\begin{itemize}
    %% \item add presentation as cite,
    \item Review MLCommons \cite{www-mlcommons} 
    \item Use jabref \cite{www-jabrefg-org} for citation management
    \item Frequently check github \cite{www-mlcommons-eathquake}
    \item Read up on TFT \cite{www-onnen2021}
    \item Become familiar with the Attention Paper \cite{vaswani2017attention}
    \item Learn how to selfdeploy jupyterlab \cite{www-jupyterlab}

    \item Begin to learn about papermill \cite{www-papermill}
      \TODO{Rivanna's core configurations might be a bit limited on
        this front.  They do provide multiple versions of software
        (such as py2.7 and py3.8), but we may need to get permission
        to do something in userspace if we need to be very specific on
        a version of python.  Do we want to look into using tox,
        conda, or leave this up to the container ecosystem to solve?}

      \TODO{Gregor: add how to use conda and modules to switch python versions}
    \item virtualenv on rivanna for a particular python version.
    \item depends on Tensorflow
      
    \TODO{Below this line are objectives or targets to take the
      current modeling solution and mature it for other platforms /
      ecosystems.  Rivanna uses Lua's lmod ecosystem for jailing a
      process, and anaconda uses solved environments for dependencies
      that extend beyond just python modules.}

  \item Familiarize yourself with Rivanna's modules \cite{www-modules}  and conda \cite{www-conda} environments.
    \item pytorch \cite{www-pytorch}
    \TODO{There is interest in comparing pytorch and tensoflow}
    \item horovod \cite{www-horovod}

      \TODO{MLCommons project that can target a few platforms using a
        YAML contract.  Once we solve the target environment, we can
        likely target porting this way.}
    
    \item \mlcube{} \cite{www-mlcube} 
\end{itemize}


\subsection{Generated List}

To find a general list of todo actions, consult

\begin{itemize}
\item The \href{https://github.com/cybertraining-dsc/capstone-eartquake/blob/main/TODO.md}{TODO.md} 
\item The \href{https://github.com/Data-ScienceHub/mlcommons-science/projects/1}{GitHub Project}
\end{itemize}

\section{Progress Reports}

\subsection{Progress Report A}

Dates covered: Jan 27 - Feb 23rd

Due: Feb 24th, 2022

\begin{itemize}
\item Reviewed background information \GITISSUE{1}
    \begin{itemize}
    \item Earthquake Nowcasting with Deep Learning
    \item MLCommons Benchmark Presentation
    \item Attention is ALl you Need
    \item MLCommons TEvolOp Earthquake Forcasting objectives
    \item Learning on Temporal Fusion Transformers
    \item MLCommons and AI for Science illustrated by Deep Learning for Geospatial Time Series
    \end{itemize}
\item Reviewed background technologies \GITISSUE{2}
    \begin{itemize}
    \item lmod - lua module and environment management (used on rivanna)
    \item pip / conda env - python package management
    \item SLURM - an HPC batch job runner
    \item papermill - parameterized jupyter notebooks
    \item horovod - distributed framework for running multiple deep learning frameworks
    \item \mlcube{} - MLCommons repeatable machine learning framework with multiple backends.
    \item singularity - container runtimes targeting HPC clusters
    \item docker
\end{itemize}
\item Wrote introduction to paper (in progress) \footnote{\url{https://www.overleaf.com/project/61f1e28f076f4111c3ba927e}}
    \begin{itemize}
    \item added references to references.bib in overleaf
    \end{itemize}

  \item Revised Intro to Python guide for conda installations so it
    doesn't interfere with pyenv or other shell
    commands.\footnote{\url{https://github.com/cloudmesh-community/book/pull/574}} \footnote{\url{https://github.com/cloudmesh-community/book/pull/575}}

  \item Solved conda environment and requirements.txt for colab notebook

    \GITISSUE{12},
    \GITPULL{1},
    \GITPULL{3}.
  
\item build docker container image that can run python 3.9.7 and conda
\item verified access to GPU allocations in Rivanna

\end{itemize}

\subsection{Progress Report B}

Dates covered: Feb 24th - Mar 23rd

Due: Mar 24th, 2022

\subsection{Presentation}

Dates covered: Mar 24th - April 27th

Due: April 28th, 2022

\clearpage


\listoftodos{}


\end{document}
\endinput
%%
%% End of file `sample-sigplan.tex'.
